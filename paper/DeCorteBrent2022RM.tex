%==============================================================================
% Sjabloon onderzoeksvoorstel bachelorproef
%==============================================================================%
% Compileren in TeXstudio:
%
% - Zorg dat Biber de bibliografie compileert (en niet Biblatex)
%   Options > Configure > Build > Default Bibliography Tool: "txs:///biber"
% - F5 om te compileren en het resultaat te bekijken.
% - Als de bibliografie niet zichtbaar is, probeer dan F5 - F8 - F5
%   Met F8 compileer je de bibliografie apart.
%
% Als je JabRef gebruikt voor het bijhouden van de bibliografie, zorg dan
% dat je in ``biblatex''-modus opslaat: File > Switch to BibLaTeX mode.

\documentclass{hogent-article}
\usepackage{graphicx}
\usepackage{lipsum} % Voor vultekst


%------------------------------------------------------------------------------
% Metadata over het artikel
%------------------------------------------------------------------------------

%---------- Titel & auteur ----------------------------------------------------

% TODO: (fase 2) geef werktitel van je eigen voorstel op
\PaperTitle{machine learning for the detection of sleep bruxism}
% Dit is typisch de opdracht en het vak waarvoor dit artikel geschreven is, bv.
% ``Verslag onderzoeksproject Onderzoekstechnieken 2018-2019''
\PaperType{Paper Research Methods: onderzoeksvoorstel}

% TODO: (fase 1) vul je eigen naam in als auteur, geef ook je emailadres mee!
\Authors{Brent De Corte\textsuperscript{1}} % Authors

% Als het hier effectief gaat om een voorstel voor de bachelorproef, dan ben je
% hier verplicht de naam van je co-promotor in te vullen. Zoniet, dan kan je het
% leeg laten.
\CoPromotor{}

% Contactinfo: Geef hier de contactgegevens van elke auteur van het artikel (en
% indien van toepassing ook van de co-promotor).
\affiliation{
  \textsuperscript{1} \href{mailto:brent.decorte@student.hogent.be}{brent.decorte@student.hogent.be}}


%---------- Abstract ----------------------------------------------------------

\Abstract{% TODO: (fase 6)
This research will be focused on the creation of an audio detection Deep Learning model integrated in a mobile application to detect sleep bruxism at home.
The model will work by transforming the audio samples into spectrograms with additional frequency and time masking as to make it an image classification problem . The model, when finished, will be implemented in an application using the python framework Kivy.}

%---------- Onderzoeksdomein en sleutelwoorden --------------------------------
% TODO: (fase 2) Vul de sleutelwoorden aan.

% Het eerste sleutelwoord beschrijft het onderzoeksdomein. Je kan kiezen uit
% deze lijst:
%
% - Mobiele applicatieontwikkeling
% - Webapplicatieontwikkeling
% - Applicatieontwikkeling (andere)
% - Systeembeheer
% - Netwerkbeheer
% - Mainframe
% - E-business
% - Databanken en big data
% - Machineleertechnieken en kunstmatige intelligentie
% - Andere (specifieer)
%
% De andere sleutelwoorden zijn vrij te kiezen.

\Keywords{Machineleertechnieken en kunstmatige intelligentie; sound detection; Bruxism}
\newcommand{\keywordname}{Sleutelwoorden} % Defines the keywords heading name

%---------- Titel, inhoud -----------------------------------------------------

\bibliography{bibliografie}

\begin{document}



\flushbottom % Makes all text pages the same height
\maketitle % Print the title and abstract box
\tableofcontents % Print the contents section
\thispagestyle{empty} % Removes page numbering from the first page

%------------------------------------------------------------------------------
% Hoofdtekst
%------------------------------------------------------------------------------

\section{Inleiding}

% TODO: (fase 2) introduceer je gekozen onderwerp, formuleer de onderzoeksvraag en deelvragen. Wat is de doelstelling (is die S.M.A.R.T.?), wat zal het resultaat zijn van het onderzoek (een Proof-of-Concept, een prototype, een advies, ...)? Waarom is het nuttig om dit onderwerp te onderzoeken?

Sleep Bruxism (SB) is the involuntary grinding and clenching of teeth during sleep.
SB can lead to an array of issues such as hypertension headaches, dental issues including surface loss of the teeth and fractures and the complaints of partners in life.
These issues may lead to a decrease in quality of life, therefore it is important to quickly detect if you suffer from SB .
\bigbreak
While there are methods of detecting SB at home using jaw musculature measurements, such as Bitestrip () .
Although they are adequate to detect SB, they are shortcomings, such as the cost and limitations of application of the device.
As there is currently no way to easily detect if you suffer from SB at home that is cheap, reliable and doesn’t interfere with the process of sleeping
To conclude, there is value in the creation of a system using machine learning models to detect SB that is easy to access and use \autocite{Wei_2020} .



\section{Overzicht literatuur}

% TODO: (fase 4) schrijf de literatuurstudie uit en gebruik waar gepast referenties naar de vakliteratuur.

% Refereren naar de literatuur kan met:
% \autocite{BIBTEXKEY} -> (Auteur, jaartal)
% \textcite{BIBTEXKEY} -> Auteur (jaartal)
%Voorbeeld van een referentie waar de auteursnaam geen onderdeel van de zin is~\autocite{Moore2002}.

Currently devices do exist for the purpose of detecting SB through the use of electromyography, which transforms electrical signals from nerves to something interpretable,  and the  heartrate which increases before the onset of a SB event . (\cite{Deregibus_2013})  .
\bigbreak
Regarding data augmentation methods exist to expand the original dataset and or altering the original samples following the ideas mentioned in() .\newline
\bigbreak

\begin{itemize}
	\item Time streching
	\bigbreak 
	\begin{itemize}
		\item Altering the tempo of the audio so the length of the sample increases or decreases .
		Zero padding or cropping can be used to fit it to the original model inputsize .
	\end{itemize}
\bigbreak
		\item Adding noise
		\bigbreak
	\begin{itemize}
		\item The addition of Gaussian noise with a hyperparameter that determines the amplitude in the range E [0.001 , 0.015] according to () can be used to reduce overfitting .
		The use of noise is not only limited to the input but it's use can also be extended to the other components of a neural network such as the activation functions . 
	\end{itemize}
\bigbreak
	\item pitch shift
	\bigbreak 
\begin{itemize}
	\item Altering the pitch of the sample by n semitones just as a time strech does on the time-scale .
	Semitones being the smallest step up or down for a given pitch .
	
\end{itemize}
\bigbreak
	\item SpecAugment
	\bigbreak
\begin{itemize}
	\item The dual components of SpecAugment are the masking of time and the frequencies of the spectrograms as to make the model better resistant against loss of information .
\end{itemize}
\end{itemize}






\bigbreak




On the front of audio detection 
Other audio detection mechanisms do exist  there are applications to detect sound, one of those being a cough detector.
The cough detector () uses a convolutional model where 1 second intervals of their sound recording are transformed with a Fourier transform to decompose the audio signal in it's constituant parts and a Hann window to create spectrograms to classify.


\section{Methodologie}

% TODO: (fase 5) beschrijf in detail in welke fasen je onderzoek uiteenvalt, hoe lang elke fase duurt en wat het concrete resultaat van elke fase is. Welke onderzoekstechniek ga je toepassen om elk van je onderzoeksvragen te beantwoorden? Gebruik je hiervoor experimenten, vragenlijsten, simulaties? Je beschrijft ook al welke tools je denkt hiervoor te gebruiken of te ontwikkelen.
\subsection{Observation}

The necessary requirements are the collection, processing and labeling of data .
The collection of the data will be done through recordings taken from a mobile phone as to best replicate the use in real life .

\begin{figure}[h!]
    \centering
    \includegraphics[width=0.7\linewidth]{../../../Afbeeldingen/milspec_brux}
    \caption{mil spectrogram of teeth grinding (duration 3s/stereo)}
    \label{fig:mil spectrogram of teeth grinding (duration 3s)}
\end{figure}


Data collection through subjects can be amplified with the use of 
time shifting, pitch shifting and adding noise .
\newline
Any data collected will be stored using MongoDB in it's processed form as to ensure the privacy of the subjects.

\bigbreak
When the data has been collected and augmented it can be coverted to individual slices .
With the data seperated and labeled spectrograms can be made that allows the problem to be solved as an image classification problem .
\bigbreak
With the spectrograms obtained, time and frequency masking can be used whereby we block out ranges of time in the former and frequencies in the latter to augment the existing spectrograms and make the model more robust to missing info at the end .
\bigbreak
When the data has been processed a working model can be created . The selection of the model in question will be based on factors like recall and precision . 
When the working model has been chosen and created, the model can be implemented in a proof of concept .
most of the amount of time accorded to this step will be accorded to data collection .

\subsection{proof of concept}

The proof of concept will be built using the python framework Kivy to build a mobile application based on the ease-of-use across platforms .

To summarize, data will be collected and processed from SB and non SB subjects.
A working model will be created based on the processed data that can be implemented in a POC mobile application .



\section{Verwachte conclusies}

% TODO: (fase 6) beschrijf wat je verwacht uit je onderzoek en waarom (bv. volgens je literatuuronderzoek is softwarepakket A het meest gebruikte en denk je dat het voor deze casus ook het meest geschikt zal zijn). Natuurlijk kan je niet in de toekomst kijken en mag je geen alternatieve mogelijkheden uitsluiten. In de praktijk gebeurt het ook vaak dat een onderzoek tot verrassende resultaten leidt, dat maakt het proces nog interessanter!

The creation of a model, most likely a convolutional network, that is able to detect the symptom of grinding of teeth that is linked to sleep bruxism and a mobile application that can incorporate the model built using the python framework Kivy .




%------------------------------------------------------------------------------
% Referentielijst
%------------------------------------------------------------------------------
% TODO: (fase 4) de gerefereerde werken moeten in BibTeX-bestand
% bibliografie.bib voorkomen. Gebruik JabRef om je bibliografie bij te
% houden.

\phantomsection
\printbibliography

\end{document}
